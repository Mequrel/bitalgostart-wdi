\documentclass[
fontsize=12pt %
,english %
,headinclude %
,headsepline % line between head an document text
%,BCOR=12mm %
]{scrbook} % twosided, A4 paper
\usepackage[T1]{fontenc}
\usepackage{polski}
\usepackage[utf8]{inputenc}
\usepackage[french,polutonikogreek,polish]{babel}
%\geometry{verbose,a4paper,tmargin=3cm,bmargin=2cm,lmargin=2cm,rmargin=2cm}
\usepackage{blindtext} % provides blindtext with sectioning
\usepackage{scrpage2} % header and footer for KOMA-Script
\usepackage{graphicx}
\clearscrheadfoot % deletes header/footer
\pagestyle{scrheadings} % use following definitions for header/footer
% definitions/configuration for the header
\rehead[]{\Large \textbf{BitAlgo Start}} % equal page, right position (inner)
\lohead[]{\Large \textbf{BitAlgo Start}} % odd page, left position (inner)
\cehead[]{Zadanie 1:\\Odwr�� list� jednokierunkow�}
\cohead[]{Zadanie 1:\\Odwr�� list� jednokierunkow�}
\lehead[]{\includegraphics[width=15mm]{logo.png}} % equal page, left (outer) position
\rohead[]{\includegraphics[width=15mm]{logo.png}}
% definitions/configuration for the footer
\cofoot[\pagemark]{\pagemark} % odd page, center position
\begin{document}
\vspace{50 mm}
\hspace{50 mm}
\newline
\par{\Large \textbf{Zadanie 1: Odwr�� list� jednokierunkow�}} \\ \\
Twoim zadaniem jest zaimplementowanie funkcji Node * reverse(Node *head), kt�ra odwraca zadan� list� jednokierunkow�.
\\ \\
\par{\Large \textbf{Format wej�cia}} \\ \\
Pierwsza linia wej�cia zawiera liczb� ca�kowit� $n$, $1 \leq n \leq 2^{15}$ - ilo�� element�w listy. W kolejnej linii znajduje si� dok�adnie $n$ liczb oddzielonych spacj�, kt�re stanowi� warto�ci element�w listy. Liczby te s� z przedzia�u $[-2^{15}; 2^{15}]$.
\\ \\
\par{\Large \textbf{Format wyj�cia}} \\ \\
Na wyj�ciu powinny zosta� wypisane warto�ci listy w odwrotnej kolejno�ci. Ka�dy element powinien by� oddzielony spacj�.
\\ \\
\par{\Large \textbf{Przyk�ad}} \\ \\
\begin{tabular}{ p{7cm} p{7cm} }
Dla danych wej�ciowych: \hspace{40mm}& Poprawn� odpowiedzi� jest \\
& \\
% define input here
10 \newline
9 8 11 20 3 1 9 5 9 10 \newline
&
% define output here
10 9 5 9 1 3 20 11 8 9 \newline
\\
\end{tabular}
\end{document}